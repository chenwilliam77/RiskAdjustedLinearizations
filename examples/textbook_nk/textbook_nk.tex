\documentclass[12 pt, oneside]{article}
\textheight 9 in
\textwidth 6.5 in
\topmargin 0 in
\oddsidemargin .3 in
\evensidemargin .3 in
\usepackage{amssymb}
\usepackage{amsmath,amsthm}
\usepackage{amsbsy,paralist}
\usepackage{appendix}
\usepackage{natbib}
\usepackage{amsfonts}
\usepackage{graphicx}
\usepackage{epsfig}
\usepackage{color}
\usepackage{mathrsfs}
\usepackage{fancyhdr}
\usepackage{setspace}
%\usepackage[nodisplayskipstretch]{setspace}
\usepackage{fullpage}
\usepackage{cancel}
\usepackage{pgfplots}
\usepackage{lipsum}
% \usepackage{subfig}
\usepackage{wrapfig,subcaption}
\usepackage{xfrac}
\usepackage{bbm}
\let\oldemptyset\emptyset
\let\emptyset\varnothing
\newtheorem*{thm}{Theorem}
\newtheorem*{lem}{Lemma}
\newtheorem{lemma}{Lemma}[section]
\newtheorem{lemnum}{Lemma}
\newtheorem*{cor}{Corollary}
\newtheorem{corollary}{Corollary}[section]
\newtheorem{cornum}{Corollary}
\newtheorem{theorem}{Theorem}[section]
\newtheorem{thmnum}{Theorem}
\newtheorem*{prop}{Proposition}
\newtheorem{proposition}{Proposition}[section]
\newtheorem{propnum}{Proposition}
\theoremstyle{definition}
\newtheorem*{remark}{Remarks}
\theoremstyle{definition}
\newtheorem*{eg}{Example}
\theoremstyle{definition}
\newtheorem*{defn}{Definition}
\newtheorem{definition}{Definition}
\newtheorem{defnnum}{Definition}[section]
\newcommand{\bigsum}[2]{\sum\limits_{#1}^{#2}}
\newcommand{\bigprod}[2]{\prod\limits_{#1}^{#2}}
\newcommand{\nlim}{\lim_{n\ra\infty}}
\DeclareMathOperator{\as}{a.s.}
\DeclareMathOperator{\almalw}{a.a.}
\newcommand{\vecx}{\vec{x}}
\newcommand{\vecy}{\vec{y}}
\DeclareMathOperator{\Char}{Char}
\DeclareMathOperator{\orb}{orb}
\DeclareMathOperator{\stab}{stab}
\DeclareMathOperator{\Aut}{Aut}
\DeclareMathOperator{\Inn}{Inn}
\DeclareMathOperator{\lcm}{lcm}
\DeclareMathOperator{\card}{card}
\DeclareMathOperator{\Cl}{Cl}
\DeclareMathOperator{\Int}{Int}
\DeclareMathOperator{\var}{Var}
\DeclareMathOperator{\cov}{Cov}
\DeclareMathOperator{\io}{i.o.}
\DeclareMathOperator{\sgn}{sgn}
\DeclareMathOperator{\tr}{trace}
\DeclareMathOperator{\diag}{diag}
\DeclareMathOperator{\vect}{vec}
\DeclareMathOperator{\diver}{div}
\DeclareMathOperator{\gradi}{grad}
\newcommand{\norm}[1]{\left\lVert#1\right\rVert}
\newcommand{\bfSigma}{\mathbf{\Sigma}}
\newcommand{\bfc}{\mathbf{c}}
\newcommand{\bfb}{\mathbf{b}}
\newcommand{\bfa}{\mathbf{a}}
\newcommand{\bfd}{\mathbf{d}}
\newcommand{\bff}{\mathbf{f}}
\newcommand{\bfh}{\mathbf{h}}
\newcommand{\bfg}{\mathbf{g}}
\newcommand{\bfi}{\mathbf{i}}
\newcommand{\bfj}{\mathbf{j}}
\newcommand{\bfk}{\mathbf{k}}
\newcommand{\bfl}{\mathbf{l}}
\newcommand{\bfm}{\mathbf{m}}
\newcommand{\bfn}{\mathbf{n}}
\newcommand{\bfo}{\mathbf{o}}
\newcommand{\bfp}{\mathbf{p}}
\newcommand{\bfq}{\mathbf{q}}
\newcommand{\bfr}{\mathbf{r}}
\newcommand{\bfs}{\mathbf{s}}
\newcommand{\bft}{\mathbf{t}}
\newcommand{\bfx}{\mathbf{x}}
\newcommand{\bfy}{\mathbf{y}}
\newcommand{\bfz}{\mathbf{z}}
\newcommand{\bfu}{\mathbf{u}}
\newcommand{\bfv}{\mathbf{v}}
\newcommand{\bfw}{\mathbf{w}}
\newcommand{\bfX}{\mathbf{X}}
\newcommand{\bfY}{\mathbf{Y}}
\newcommand{\bfA}{\mathbf{A}}
\newcommand{\bfB}{\mathbf{B}}
\newcommand{\bfC}{\mathbf{C}}
\newcommand{\bfD}{\mathbf{D}}
\newcommand{\bfE}{\mathbf{E}}
\newcommand{\bfF}{\mathbf{F}}
\newcommand{\bfG}{\mathbf{G}}
\newcommand{\bfH}{\mathbf{H}}
\newcommand{\bfI}{\mathbf{I}}
\newcommand{\bfJ}{\mathbf{J}}
\newcommand{\bfK}{\mathbf{K}}
\newcommand{\bfL}{\mathbf{L}}
\newcommand{\bfU}{\mathbf{U}}
\newcommand{\bfV}{\mathbf{V}}
\newcommand{\bfW}{\mathbf{W}}
\newcommand{\bfZ}{\mathbf{Z}}
\newcommand{\bfM}{\mathbf{M}}
\newcommand{\bfN}{\mathbf{N}}
\newcommand{\bfQ}{\mathbf{Q}}
\newcommand{\bfO}{\mathbf{O}}
\newcommand{\bfR}{\mathbf{R}}
\newcommand{\bfS}{\mathbf{S}}
\newcommand{\bfT}{\mathbf{T}}
\newcommand{\bfP}{\mathbf{P}}
\newcommand{\bfzero}{\mathbf{0}}
\newcommand{\R}{\mathbb{R}}
\newcommand{\E}{\mathbb{E}}
\newcommand{\N}{\mathbb{N}}
\newcommand{\Q}{\mathbb{Q}}
\newcommand{\Z}{\mathbb{Z}}
\newcommand{\curA}{\mathscr{A}}
\newcommand{\curC}{\mathscr{C}}
\newcommand{\curD}{\mathscr{D}}
\newcommand{\curE}{\mathscr{E}}
\newcommand{\curH}{\mathscr{H}}
\newcommand{\curJ}{\mathscr{J}}
\newcommand{\curK}{\mathscr{K}}
\newcommand{\curN}{\mathscr{N}}
\newcommand{\curO}{\mathscr{O}}
\newcommand{\curQ}{\mathscr{Q}}
\newcommand{\curS}{\mathscr{S}}
\newcommand{\curT}{\mathscr{T}}
\newcommand{\curU}{\mathscr{U}}
\newcommand{\curV}{\mathscr{V}}
\newcommand{\curW}{\mathscr{W}}
\newcommand{\curZ}{\mathscr{Z}}
\newcommand{\curI}{\mathscr{I}}
\newcommand{\curB}{\mathscr{B}}
\newcommand{\curF}{\mathscr{F}}
\newcommand{\curG}{\mathscr{G}}
\newcommand{\curM}{\mathscr{M}}
\newcommand{\curL}{\mathscr{L}}
\newcommand{\curP}{\mathscr{P}}
\newcommand{\curR}{\mathscr{R}}
\newcommand{\curX}{\mathscr{X}}
\newcommand{\curY}{\mathscr{Y}}
\newcommand{\calA}{\mathcal{A}}
\newcommand{\calB}{\mathcal{B}}
\newcommand{\calC}{\mathcal{C}}
\newcommand{\calD}{\mathcal{D}}
\newcommand{\calE}{\mathcal{E}}
\newcommand{\calF}{\mathcal{F}}
\newcommand{\calG}{\mathcal{G}}
\newcommand{\calH}{\mathcal{H}}
\newcommand{\calI}{\mathcal{I}}
\newcommand{\calJ}{\mathcal{J}}
\newcommand{\calL}{\mathcal{L}}
\newcommand{\calK}{\mathcal{K}}
\newcommand{\calM}{\mathcal{M}}
\newcommand{\calN}{\mathcal{N}}
\newcommand{\calO}{\mathcal{O}}
\newcommand{\calP}{\mathcal{P}}
\newcommand{\calQ}{\mathcal{Q}}
\newcommand{\calR}{\mathcal{R}}
\newcommand{\calS}{\mathcal{S}}
\newcommand{\calT}{\mathcal{T}}
\newcommand{\calU}{\mathcal{U}}
\newcommand{\calV}{\mathcal{V}}
\newcommand{\calW}{\mathcal{W}}
\newcommand{\calX}{\mathcal{X}}
\newcommand{\calY}{\mathcal{Y}}
\newcommand{\calZ}{\mathcal{Z}}
\newcommand{\RA}{\Rightarrow}
\newcommand{\ra}{\rightarrow}
\newcommand{\fd}{\vspace{2.5mm}}
\newcommand{\ds}{\vspace{1mm}}
\setlength{\parindent}{0pt}
\begin{document}
These notes for the New Keynesian model follow Eric Sims's notes. Section \ref{sec:model} solves the equilibrium conditions of the New Keynesian model, and Section \ref{sec:ral} transforms the equilibrium conditions into the desired form for a risk-adjusted linearization.

\section{Model}\label{sec:model}


\subsection{Household}

Households solve the problem
\begin{align*}
  \max_{C_t, N_t, B_{t + 1}, M_t} \E_0 \sum_{t = 0}^\infty \beta^t \left( \frac{C_t^{1 - \sigma}}{1 - \sigma} - \psi \frac{N_t^{1 + \eta}}{1 + \eta} + \theta\log\left(\frac{M_t}{P_t}\right) \right)
\end{align*}
subject to the budget constraint
\begin{align*}
  P_tC_t + B_{t + 1} + M_t - M_{t - 1} \leq W_t N_t + \Pi_t + (1 + i_{t - 1})B_t.
\end{align*}
In this model, households have demand for money $M_t$, which is also the numeraire. The price of goods in terms of money is $P_t$.
The stock of nominal bonds a households has is $B_t$.
Note that $B_t$ will be pre-determined at period $t$ while $M_t$ will not be ($M_{t - 1}$ is pre-determined).

The Lagrangian for the household is
\begin{align*}
\calL & = \E_0 \sum_{t = 0}^\infty \beta^t\left[\frac{C_t^{1 - \sigma}}{1 - \sigma} - \psi \frac{N_t^{1 + \eta}}{1 + \eta} + \theta\log\left(\frac{M_t}{P_t}\right)\right]\\
        &\quad + \E_0 \sum_{t = 0}^\infty \beta^t\left[\lambda_t( W_t N_t + \Pi_t + (1 + i_{t - 1})B_t-  P_tC_t - B_{t + 1} - (M_t - M_{t - 1}))\right],
\end{align*}
which implies first-order conditions
\begin{align*}
  0 & = C_t^{- \sigma} - \lambda_t P_t\\
  0 & = -\psi N_t^\eta + \lambda_t W_t\\
  0 & = -\lambda_t + \beta \E_t\lambda_{t + 1}(1 + i_t)\\
  0 & = \theta \frac{1}{M_t} - \lambda_t + \beta\E_t\lambda_{t + 1}.
\end{align*}
The first two equations can be combined by isolating $\lambda_t$.
Using $\lambda_t = C_t^{-\sigma} / P_t$, we can obtain the Euler equation for households
and an equation relating money balances to consumption.
\begin{align*}
  C_t^{-\sigma} \frac{W_t}{P_t} & = \psi N_t^\eta,\\
  C_t^{-\sigma} & = \beta\E_t C_{t + 1}^{-\sigma} ( 1 + i_t),\\
  \theta\left(\frac{M_t}{P_t}\right)^{-1} & = \frac{i_t}{1 + i_t}C_t^{-\sigma}.
\end{align*}

\subsection{Production}

\paragraph{Final Producers}

There is a representative final goods firm which sells consumption goods in a competitive market.
It aggregates intermediate goods using the CES technology
\begin{align*}
  Y_t & = \left(\int_0^1 Y_t(j)^{ \frac{\epsilon - 1}{\epsilon}}\right)^{\frac{\epsilon}{\epsilon - 1}}
\end{align*}
where $\epsilon > 1$ so that inputs are substitutes. Profit maximization for the final good firm is
\begin{align*}
  \max_{Y_t(j)} P_t\left(\int_0^1 Y_t(j)^{ \frac{\epsilon - 1}{\epsilon}}\right)^{\frac{\epsilon}{\epsilon - 1}} - \int_0^1 P_t(j) Y_t(j)\, dj.
\end{align*}
The FOC for $Y_t(j)$ is
\begin{align*}
  0 & = P_t\frac{\epsilon}{\epsilon - 1}\left(\int_0^1 Y_t(j)^{\frac{\epsilon}{\epsilon - 1}}\right)^{\frac{1}{\epsilon - 1}}\frac{\epsilon - 1}{\epsilon} Y_t(j)^{-\frac{1}{\epsilon}} - P_t(j)\\
  0 & = \left(\int_0^1 Y_t(j)^{\frac{\epsilon}{\epsilon - 1}}\right)^{\frac{1}{\epsilon - 1}}Y_t(j)^{-\frac{1}{\epsilon}} - \frac{P_t(j)}{P_t}\\
  0 & = \left(\int_0^1 Y_t(j)^{\frac{\epsilon}{\epsilon - 1}}\right)^{-\frac{\epsilon}{\epsilon - 1}}Y_t(j) - \left(\frac{P_t(j)}{P_t}\right)^{-\epsilon}\\
  Y_t(j) & =  \left(\frac{P_t(j)}{P_t}\right)^{-\epsilon} Y_t.
\end{align*}
Plugging this quantity into the identity
\begin{align*}
  P_tY_t & = \int_0^1 P_t(j) Y_t(j)\, dj
\end{align*}
and simplifying yields the price index
\begin{align*}
  P_t & = \left(\int_0^1 P_t(j)^{1 - \epsilon}\, dj\right)^{\frac{1}{1 - \epsilon}}.
\end{align*}

\paragraph{Intermediate Producers}

Intermediate goods are producing according to the linear technology
\begin{align*}
  Y_t(j) = A_t N_t(j).
\end{align*}
Intermediate producers minimize cost subject to the constraint of meeting demand and Calvo price rigidities. Formally,
\begin{align*}
  \min_{N_t(j)} W_t N_t(j)\quad\quad\quad\text{s.t.}\quad\quad\quad A_t N_t(j) \geq \left(\frac{P_t(j)}{P_t}\right)^{-\epsilon}Y_t.
\end{align*}
The Lagrangian is
\begin{align*}
  \calL = W_tN_t(j) + \varphi_t(j)\left(\left(\frac{P_t(j)}{P_t}\right)^{-\epsilon}Y_t - A_t N_t(j)\right),
\end{align*}
so the first-order condition is
\begin{align*}
  0 & = W_t - \varphi_t(j) A_t \RA \varphi_t(j) = \frac{W_t}{A_t}.
\end{align*}
The multiplier $\varphi_t$ can be interpreted as the nominal marginal cost. Let $mc_t$ be the real marginal cost. Then profits for
an intermediate producer is
\begin{align*}
  \Pi_t(j) = \frac{P_t(j)}{P_t}Y_t(j) - mc_t Y_t(j).
\end{align*}

In addition to the labor choice, firms also have the chance to reset prices in every period with probability $1 - \phi$. This problem can be
written as
\begin{align*}
  \max_{P_t(j)} \E_t\sum_{s = 0}^\infty(\beta \phi)^s \frac{u'(C_{t + s})}{u'(C_t)}\left(\frac{P_t(j)}{P_{t + s}}\left(\frac{P_t(j)}{P_{t + s}}\right)^{-\epsilon}Y_{t + s} - mc_{t + s}\left(\frac{P_t(j)}{P_{t + s}}\right)^{-\epsilon}Y_{t + s}\right),
\end{align*}
where I have imposed that output equals demand. The first-order condition is
\begin{align*}
0 & =  (1 - \epsilon)P_t(j)^{-\epsilon}\E_t\sum_{s = 0}^\infty (\beta\phi)^s \frac{u'(C_{t + s})}{u'(C_t)}(P_{t + s})^{-(1 - \epsilon)}Y_{t + s}\\
  &\quad + \epsilon P_t(j)^{-\epsilon - 1}\E_t\sum_{s = 0}^\infty (\beta\phi)^s \frac{u'(C_{t + s})}{u'(C_t)} mc_{t + s}P_{t + s}^{\epsilon}Y_{t + s}
\end{align*}
Divide by $P_t(j)^{-\epsilon} / u'(C_t)$ and re-arrange to obtain
\begin{align*}
  P_t(j) & = \frac{\epsilon}{\epsilon - 1}\frac{\E_t\sum_{s = 0}^\infty (\beta\phi)^s u'(C_{t + s}) mc_{t + s}P_{t + s}^{\epsilon}Y_{t + s}}{\E_t\sum_{s = 0}^\infty (\beta\phi)^s u'(C_{t + s})P_{t + s}^{\epsilon - 1}Y_{t + s}}.
\end{align*}
This expression gives the optimal reset price $P_t^*$, which we can write more compactly as
\begin{align*}
  P_t^* & = \frac{\epsilon}{\epsilon - 1}\frac{X_{1, t}}{X_{2, t}}
\end{align*}
where
\begin{align*}
  X_{1, t} & = u'(C_t) mc_t P_t^{\epsilon} Y_t + \phi \beta \E_tX_{1, t + 1}\\
  X_{2, t} & = u'(C_t) P_t^{\epsilon - 1} Y_t + \phi \beta \E_tX_{2, t + 1}.
\end{align*}

\subsection{Equilibrium and Aggregation}

To close the model, I assume that the log of technology $A_t$ follows the AR(1)
\begin{align*}
  \log A_t = \rho_a \log A_{t _ 1} + \varepsilon_{a, t},
\end{align*}
and the growth rate in the log money supply follows the AR(1)
\begin{align*}
  \Delta \log M_t & = (1 - \rho_m) \pi + \rho_m \Delta \log M_{t - 1} + \varepsilon_{m, t},
\end{align*}
where $\pi$ is the steady-state rate of inflation. Note that this specification ensures
that money balances grow at the same rate as the price level, which ensures real balances
are stationary. To re-write the money growth equation in real terms, note that
\begin{align*}
  \log(m_t) = \log(M_t) - \log(P_t) \RA \Delta \log(m_t) = \log(m_t) - \log(m_{t - 1}) = \Delta \log(M_t) - \log(1 + \pi_t),
\end{align*}
hence
\begin{align*}
  \Delta \log(m_t) & = (1 - \rho_m) \pi + \rho_m \Delta \log(m_{t - 1}) + \rho_m \log(1 + \pi_{t - 1}) - \log(1 + \pi_t) +  \varepsilon_{m, t}.
\end{align*}

In equilibrium, bond-holding must be zero, hence
\begin{align*}
  C_t = w_t N_t + \frac{\Pi_t}{P_t}.
\end{align*}
Real dividends $\Pi_t$ satisfy the accounting identity
\begin{align*}
  \frac{\Pi_t}{P_t} & = \int_0^1\left(\frac{P_t(j)}{P_t}Y_t(j) - \frac{W_t}{P_t}N_t(j)\right)\, dj\\
                    & = \int_0^1\frac{P_t(j)}{P_t}Y_t(j) \, dj - w_t\int_0^1N_t(j)\, dj
\end{align*}
where $w_t = W_t / P_t$. Aggregate labor supply $N_t$ equals aggregate labor demand in equilibrium, and
market-clearing for consumption requires
\begin{align*}
  C_t = \int_0^1\frac{P_t(j)}{P_t} Y_t(j)\, dj = \int_0^1 \frac{P_t(j)}{P_t}\left(\frac{P_t(j)}{P_t}\right)^{-\epsilon} Y_t\, dj = P_t^{\epsilon - 1}Y_t\int_0^1P_t(j)^{1 - \epsilon}\, dj = Y_t
\end{align*}
since $\int_0^1 P_t(j)^{1 - \epsilon}\, dj = P_t^{1 - \epsilon}$.

The quantity $Y_t$ is aggregate output, so we must have
\begin{align*}
\int_0^1 A_t N_t(j)\, dj & = \int_0^1\left(\frac{P_t(j)}{P_t}\right)^{-\epsilon} Y_t\, dj\\
  A_tN_t & = Y_t \int_0^1\left(\frac{P_t(j)}{P_t}\right)^{-\epsilon} \, dj = v_t Y_t.
\end{align*}
Thus, aggregate output is
\begin{align*}
  Y_t = \frac{A_tN_t}{v_t}.
\end{align*}
It can be shown that $v_t \geq 1$ by applying Jensen's inequality.

Finally, recall that
\begin{align*}
  P_t^{1 - \epsilon} = \int_0^1 P_t(j)^{1 - \epsilon}\, dj.
\end{align*}
In each period, a fraction $\phi$ cannot change their price. Without loss of generality, we may re-order these firms to the top of the interval so that
\begin{align*}
  P_t^{1 - \epsilon} = (1 - \phi)(P_t^*)^{1 - \epsilon} +  \int_{1 - \phi}^1 P_{t - 1}(j)^{1 - \epsilon}\, dj.
\end{align*}
The latter term can be further simplified under the law of large numbers assumption that a positive measure of firms  which cannot change their price
still comprise a representative sample of all firms, yielding
\begin{align*}
  P_t^{1 - \epsilon} = (1 - \phi)(P_t^*)^{1 - \epsilon} +  \phi\int_0^1 P_{t - 1}(j)^{1 - \epsilon}\, dj = (1 - \phi)(P_t^*)^{1 - \epsilon} +  \phi P_{t - 1}^{1 - \epsilon}.
\end{align*}
Dividing by $P_{t - 1}^{1 - \epsilon}$ implies
\begin{align*}
  (1 + \pi_t)^{ 1 - \epsilon} & = (1 - \phi) (1 + \pi_t^*)^{1 - \epsilon} + \phi
\end{align*}
The price dispersion term can similarly be re-written in terms of aggregates by distinguishing which firms get to change prices.
\begin{align*}
  v_t & = \int_0^{1 - \phi}\left(\frac{P_t^*}{P_t}\right)^{ - \epsilon}\, dj + \int_{1 - \phi}^1 \left(\frac{P_{t - 1}(j)}{P_t}\right)^{ - \epsilon}\, dj\\
      & = \int_0^{1 - \phi}\left(\frac{P_t^*}{P_{t - 1}}\right)^{ - \epsilon}\left(\frac{P_{t - 1}}{P_t}\right)^{ - \epsilon}\, dj + \int_{1 - \phi}^1 \left(\frac{P_{t - 1}(j)}{P_{t - 1}}\right)^{ - \epsilon}\left(\frac{P_{t - 1}}{P_t}\right)^{ - \epsilon}\, dj\\
      & = (1 - \phi) ( 1  + \pi_t^*)^{-\epsilon} ( 1 + \pi_t)^{\epsilon}  +(1 + \pi_t)^{\epsilon} \int_{1 - \phi}^1 \left(\frac{P_{t - 1}(j)}{P_{t - 1}}\right)^{ - \epsilon}\,dj.
\end{align*}
By invoking the law of large assumptions applied to any positive measure subset of firms, we must have
\begin{align*}
  \int_{1 - \phi}^1 \left(\frac{P_{t - 1}(j)}{P_{t - 1}}\right)^{ - \epsilon}\,dj & = \phi\int_0^1 \left(\frac{P_{t - 1}(j)}{P_{t - 1}}\right)^{ - \epsilon}\,dj = \phi v_{t - 1}.
\end{align*}
Thus, we acquire
\begin{align*}
  v_t & = (1 - \phi) ( 1  + \pi_t^*)^{-\epsilon} ( 1 + \pi_t)^{\epsilon}  +\phi(1 + \pi_t)^{\epsilon}  v_{t - 1}\\
      & = ( 1 + \pi_t)^{\epsilon}((1 - \phi) ( 1  + \pi_t^*)^{-\epsilon}   +\phi  v_{t - 1}).
\end{align*}
To finish, we need to derive an expression characterizing $\pi_t^*$. Define
\begin{align*}
  x_{1, t} \equiv \frac{X_{1, t}}{P_t^\epsilon},\quad\quad x_{2, t} \equiv \frac{X_{2, t}}{P_t^{\epsilon - 1}}.
\end{align*}
It follows that
\begin{align*}
  x_{1, t} & = u'(C_t)mc_t Y_t + \phi \beta \E_t \frac{X_{1, t + 1}}{P_t^\epsilon}\\
           & = u'(C_t)mc_t Y_t + \phi \beta \E_t\left[ \frac{X_{1, t + 1}}{P_{t + 1}^\epsilon}\frac{P_{t + 1}^\epsilon}{P_t^\epsilon}\right]\\
           & = u'(C_t)mc_t Y_t + \phi \beta \E_t[x_{1, t + 1}(1 + \pi_{t + 1})^\epsilon]\\
  x_{2, t} & = C_t^{-\sigma} Y_t + \phi \beta \E_t\frac{X_{2, t + 1}}{P_t^{\epsilon - 1}}\\
           & = C_t^{-\sigma} Y_t + \phi \beta \E_t\left[\frac{X_{2, t + 1}}{P_{t + 1}^{\epsilon - 1}}\frac{P_{t + 1}^{\epsilon - 1}}{P_t^{\epsilon - 1}}\right]\\
           & = C_t^{-\sigma} Y_t + \phi \beta \E_t[x_{2, t + 1}(1 + \pi_{t + 1})^{\epsilon - 1}].
\end{align*}
Further,
\begin{align*}
  \frac{X_{1, t}}{X_{2, t}} = \frac{x_{1, t}}{x_{2, t}}P_t,
\end{align*}
hence
\begin{align*}
  P_t^* & = \frac{\epsilon}{\epsilon - 1}\frac{x_{1, t}}{x_{2, t}}P_t\\
  (1 + \pi_t^*) & = \frac{\epsilon}{\epsilon - 1}\frac{x_{1, t}}{x_{2, t}}(1 + \pi_t)\\
\end{align*}


All together, the full set of equilibrium conditions are
\begin{align*}
  C_t^{-\sigma} & = \beta \E_t\left[C_{t + 1}^{-\sigma}\frac{(1 + i_t)}{1 + \pi_{t + 1}}\right]\\
  C_t^{-\sigma} & = \psi\frac{N_t^\eta}{w_t}\\
  m_t & = \theta \frac{1 + i_t}{i_t} C_t^\sigma\\
  mc_t & = \frac{w_t}{A_t}\\
  C_t & = Y_t\\
  Y_t & = \frac{A_tN_t}{v_t}\\
  v_t & = ( 1 + \pi_t)^{\epsilon}((1 - \phi) ( 1  + \pi_t^*)^{-\epsilon}   +\phi  v_{t - 1})\\
  (1 + \pi_t)^{ 1 - \epsilon} & = (1 - \phi) (1 + \pi_t^*)^{1 - \epsilon} + \phi\\
  (1 + \pi_t^*) & = \frac{\epsilon}{\epsilon - 1}\frac{x_{1, t}}{x_{2, t}}(1 + \pi_t)\\
  x_{1, t} & = C_t^{-\sigma}mc_t Y_t + \phi \beta \E_t[x_{1, t + 1}(1 + \pi_{t + 1})^{\epsilon}]\\
  x_{2, t} & = C_t^{-\sigma} Y_t + \phi \beta \E_t[x_{2, t + 1}(1 + \pi_{t + 1})^{\epsilon-1}]\\
  \log A_t & = \rho_a \log(A_{t - 1}) + \varepsilon_{a, t}\\
  \Delta \log(m_t) & = (1 - \rho_m) \pi + \rho_m \Delta \log(m_{t - 1}) + \rho_m \log(1 + \pi_{t - 1}) - \log(1 + \pi_t) +  \varepsilon_{m, t}\\
  \Delta \log m_t & = \log m_t - \log m_{t - 1},
\end{align*}
which comprise 14 equations in 14 aggregate variables
$$(C_t, i_t, \pi_t, N_t, w_t, m_t, mc_t, A_t, Y_t, v_t, \pi^*_t, x_{1, t}, x_{2, t}, \Delta \log m_t).$$

Alternatively, the money growth equation can be replaced by the Taylor rule
\begin{align*}
  \log(1 + i_t) & = (1 - \rho_i) \log(1 + i) + \rho_i \log(1 + i_{t - 1}) + (1 - \rho_i)\phi_\pi (\log(1 + \pi_t) - \log( 1 + \pi))  + \varepsilon_{i, t},
\end{align*}
and the third equation relating money demand to consumption could also be ignored. To reduce the number of equations, we utilize this specification. Furthermore, we can also substitute $1 + \pi_t^*$ to remove $\pi_t^*$ from the aggregate variables.


\section{Risk-Adjusted Linearization}\label{sec:ral}

We now proceed to converting the equilibrium conditions into a suitable form for a risk-adjusted linearization. The system should conform to the representation
\begin{align*}
  0 & = \log \E_t\left[\exp\left(\xi(z_t, y_t) + \Gamma_5 z_{t + 1} + \Gamma_6 y_{t + 1}\right)\right]\\
  z_{t + 1} & = \mu(z_t, y_t) + \Lambda(z_t, y_t) (y_{t + 1} - \E_t y_{t + 1}) + \Sigma(z_t, y_t) \varepsilon_{t + 1},
\end{align*}
where $z_t$ are (predetermined) state variables and $y_t$ are (nondetermined) jump variables.

For the remainder of this section, lower case variables are the logs of previously upper case variables, and variables with a tilde are the logs of previously lower case variables (e.g. the real wage $w_t$)

\fd

The first equation becomes
\begin{align*}
  1 & = \beta \E_t\left[\frac{C_{t + 1}^{-\sigma}}{C_t^{-\sigma}}\frac{ 1 + i_t}{1 + \pi_{t + 1}}\right]\\
  0 & = \log\E_t\left[\exp\left(\log(\beta) -\sigma (c_{t + 1} - c_t) + \tilde{i}_t - \tilde{\pi}_{t + 1} \right)\right]\\
    & = \log\E_t\left[\exp\left(\underbrace{\log(\beta) + \sigma c_t + \tilde{i}_t}_{\xi} - \underbrace{\sigma c_{t + 1} - \tilde{\pi}_{t + 1}}_{\text{Forward-Looking}} \right)\right],
\end{align*}
where $c_t = \log(C_t)$, $\tilde{i}_t = \log(1 + i_t)$, and $\tilde{\pi}_t = \log(1 + \pi_{t + 1})$.

\fd

The second equation becomes
\begin{align*}
  1 & = \psi\frac{N_t^\eta}{C_t^{-\sigma} w_t}\\
  0 & = \log \E_t\left[\exp\left(\underbrace{\log(\psi) + \eta n_t - (- \sigma c_t + \hat{w}_t)}_{\xi}\right)\right],
\end{align*}
where $n_t = \log(N_t)$ and $\hat{w}_t = \log(w_t)$.

\fd

The third equation becomes
\begin{align*}
  1 & = \frac{w_t}{A_t mc_t}\\
  0 & = \log\E_t\left[\exp\left(\hat{w}_t - a_t - \tilde{mc}_t \right)\right].
\end{align*}

The fourth and fifth equation become
\begin{align*}
  0 & = \log\E_t\left[\exp\left(c_t - a_t - n_t + \hat{v}_t \right)\right].
\end{align*}

The sixth equation becomes
\begin{align*}
  0 & = \hat{v}_t - \epsilon \tilde{\pi}_t - \log((1 - \phi) \exp(\tilde{\pi}_t^*)^{ - \epsilon} + \phi \exp(\hat{v}_{t - 1})),
\end{align*}
where $\hat{v}_{t - 1}$ will be treated as an additional state variable, i.e. if $a_t = \hat{v}_t$ is a jump variable and $b_t = \hat{v}_{t - 1}$
is a state variable, then
\begin{align*}
  b_{t + 1} = a_t.
\end{align*}

The seventh equation becomes
\begin{align*}
  0 & = (1 - \epsilon)\tilde{\pi}_t - \log((1 - \phi) \exp(\tilde{\pi}_t^*)^{1 - \epsilon} + \phi).
\end{align*}

The eighth equation becomes
\begin{align*}
  0 & = \tilde{\pi}_t^* - \log\left(\frac{\epsilon}{\epsilon - 1}\right) - \tilde{\pi}_t - (\hat{x}_{1, t} - \hat{x}_{2, t})).
\end{align*}
By plugging this expression for $\tilde{\pi}_t^*$ into the previous two equations, we can also remove one more variable from the system.

\fd

The ninth and tenth equation become
\begin{align*}
  1 & = \E_t\left[\phi \beta\frac{x_{1, t + 1}(1 + \pi_{t + 1})}{x_{1, t} - C_t^{-\sigma} mc_t A_tN_t / v_t}\right]\\
  0 & = \log\E_t\left[\exp\left(\underbrace{\log(\phi) + \log(\beta)   - \log(\exp(\hat{x}_{1, t}) - \exp((1-\sigma) c_t + \tilde{mc}_t))}_{\xi} + \hat{x}_{1, t + 1} + \epsilon\tilde{\pi}_{t + 1}\right)\right]\\
  0 & = \log\E_t\left[\exp\left(\underbrace{\log(\phi) + \log(\beta)  - \log(\exp(\hat{x}_{2, t}) - \exp((1-\sigma) c_t))}_{\xi} + \hat{x}_{2, t + 1} + (\epsilon - 1)\tilde{\pi}_{t + 1}\right)\right],
\end{align*}
where the fact that $x_{1,t}$ and $x_{2, t}$ must both be positive implies $x_{1, t} - C_t^{-\sigma} mc_t Y_t$ and $x_{2, t} - C_t^{-\sigma}Y_t$ are both positive, as the expectations on the RHS are also both positive.

\fd

For the monetary policy rule, we use $\tilde{i}_{t - 1} \equiv \log(1 + i_{t - 1})$ and $\varepsilon_{i, t}$ as states and treat $i_t$ as a jump variable, hence
\begin{align*}
  \tilde{i}_t & = (1 - \rho_i) \tilde{i} + \rho_i \tilde{i}_{t - 1} + (1 - \rho_i)\phi_\pi (\tilde{\pi}_t - \tilde{\pi})  + \varepsilon_{i, t}.
\end{align*}
This formulation allows us to treat the policy rule as an expectational equation.

\fd

The above nine equations comprise the expectational equations. The following four equations comprise the states:
\begin{align*}
  a_{t + 1} & = \rho_a a_t + \varepsilon_{a, t + 1}\\
  \hat{v}_{(t - 1) + 1} & = \hat{v}_t\\
  \tilde{i}_{(t - 1) + 1} & = \tilde{i}_t\\
  \varepsilon_{i, t + 1} & = \varepsilon_{i, t + 1}
\end{align*}


To conclude, note that we have three forward difference equations. The first is the Euler equation, which can be expressed as

\end{document}