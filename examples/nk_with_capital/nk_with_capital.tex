\documentclass[12 pt, oneside]{article}
\textheight 9 in
\textwidth 6.5 in
\topmargin 0 in
\oddsidemargin .3 in
\evensidemargin .3 in
\usepackage{amssymb}
\usepackage{amsmath,amsthm}
\usepackage{amsbsy,paralist}
\usepackage{appendix}
\usepackage{natbib}
\usepackage{amsfonts}
\usepackage{graphicx}
\usepackage{epsfig}
\usepackage{color}
\usepackage{mathrsfs}
\usepackage{fancyhdr}
\usepackage{setspace}
%\usepackage[nodisplayskipstretch]{setspace}
\usepackage{fullpage}
\usepackage{cancel}
\usepackage{pgfplots}
\usepackage{lipsum}
% \usepackage{subfig}
\usepackage{wrapfig,subcaption}
\usepackage{xfrac}
\usepackage{bbm}
\let\oldemptyset\emptyset
\let\emptyset\varnothing
\newtheorem*{thm}{Theorem}
\newtheorem*{lem}{Lemma}
\newtheorem{lemma}{Lemma}[section]
\newtheorem{lemnum}{Lemma}
\newtheorem*{cor}{Corollary}
\newtheorem{corollary}{Corollary}[section]
\newtheorem{cornum}{Corollary}
\newtheorem{theorem}{Theorem}[section]
\newtheorem{thmnum}{Theorem}
\newtheorem*{prop}{Proposition}
\newtheorem{proposition}{Proposition}[section]
\newtheorem{propnum}{Proposition}
\theoremstyle{definition}
\newtheorem*{remark}{Remarks}
\theoremstyle{definition}
\newtheorem*{eg}{Example}
\theoremstyle{definition}
\newtheorem*{defn}{Definition}
\newtheorem{definition}{Definition}
\newtheorem{defnnum}{Definition}[section]
\newcommand{\bigsum}[2]{\sum\limits_{#1}^{#2}}
\newcommand{\bigprod}[2]{\prod\limits_{#1}^{#2}}
\newcommand{\nlim}{\lim_{n\ra\infty}}
\DeclareMathOperator{\as}{a.s.}
\DeclareMathOperator{\almalw}{a.a.}
\newcommand{\vecx}{\vec{x}}
\newcommand{\vecy}{\vec{y}}
\DeclareMathOperator{\Char}{Char}
\DeclareMathOperator{\orb}{orb}
\DeclareMathOperator{\stab}{stab}
\DeclareMathOperator{\Aut}{Aut}
\DeclareMathOperator{\Inn}{Inn}
\DeclareMathOperator{\lcm}{lcm}
\DeclareMathOperator{\card}{card}
\DeclareMathOperator{\Cl}{Cl}
\DeclareMathOperator{\Int}{Int}
\DeclareMathOperator{\var}{Var}
\DeclareMathOperator{\cov}{Cov}
\DeclareMathOperator{\io}{i.o.}
\DeclareMathOperator{\sgn}{sgn}
\DeclareMathOperator{\tr}{trace}
\DeclareMathOperator{\diag}{diag}
\DeclareMathOperator{\vect}{vec}
\DeclareMathOperator{\diver}{div}
\DeclareMathOperator{\gradi}{grad}
\newcommand{\norm}[1]{\left\lVert#1\right\rVert}
\newcommand{\bfSigma}{\mathbf{\Sigma}}
\newcommand{\bfc}{\mathbf{c}}
\newcommand{\bfb}{\mathbf{b}}
\newcommand{\bfa}{\mathbf{a}}
\newcommand{\bfd}{\mathbf{d}}
\newcommand{\bff}{\mathbf{f}}
\newcommand{\bfh}{\mathbf{h}}
\newcommand{\bfg}{\mathbf{g}}
\newcommand{\bfi}{\mathbf{i}}
\newcommand{\bfj}{\mathbf{j}}
\newcommand{\bfk}{\mathbf{k}}
\newcommand{\bfl}{\mathbf{l}}
\newcommand{\bfm}{\mathbf{m}}
\newcommand{\bfn}{\mathbf{n}}
\newcommand{\bfo}{\mathbf{o}}
\newcommand{\bfp}{\mathbf{p}}
\newcommand{\bfq}{\mathbf{q}}
\newcommand{\bfr}{\mathbf{r}}
\newcommand{\bfs}{\mathbf{s}}
\newcommand{\bft}{\mathbf{t}}
\newcommand{\bfx}{\mathbf{x}}
\newcommand{\bfy}{\mathbf{y}}
\newcommand{\bfz}{\mathbf{z}}
\newcommand{\bfu}{\mathbf{u}}
\newcommand{\bfv}{\mathbf{v}}
\newcommand{\bfw}{\mathbf{w}}
\newcommand{\bfX}{\mathbf{X}}
\newcommand{\bfY}{\mathbf{Y}}
\newcommand{\bfA}{\mathbf{A}}
\newcommand{\bfB}{\mathbf{B}}
\newcommand{\bfC}{\mathbf{C}}
\newcommand{\bfD}{\mathbf{D}}
\newcommand{\bfE}{\mathbf{E}}
\newcommand{\bfF}{\mathbf{F}}
\newcommand{\bfG}{\mathbf{G}}
\newcommand{\bfH}{\mathbf{H}}
\newcommand{\bfI}{\mathbf{I}}
\newcommand{\bfJ}{\mathbf{J}}
\newcommand{\bfK}{\mathbf{K}}
\newcommand{\bfL}{\mathbf{L}}
\newcommand{\bfU}{\mathbf{U}}
\newcommand{\bfV}{\mathbf{V}}
\newcommand{\bfW}{\mathbf{W}}
\newcommand{\bfZ}{\mathbf{Z}}
\newcommand{\bfM}{\mathbf{M}}
\newcommand{\bfN}{\mathbf{N}}
\newcommand{\bfQ}{\mathbf{Q}}
\newcommand{\bfO}{\mathbf{O}}
\newcommand{\bfR}{\mathbf{R}}
\newcommand{\bfS}{\mathbf{S}}
\newcommand{\bfT}{\mathbf{T}}
\newcommand{\bfP}{\mathbf{P}}
\newcommand{\bfzero}{\mathbf{0}}
\newcommand{\R}{\mathbb{R}}
\newcommand{\E}{\mathbb{E}}
\newcommand{\N}{\mathbb{N}}
\newcommand{\Q}{\mathbb{Q}}
\newcommand{\Z}{\mathbb{Z}}
\newcommand{\curA}{\mathscr{A}}
\newcommand{\curC}{\mathscr{C}}
\newcommand{\curD}{\mathscr{D}}
\newcommand{\curE}{\mathscr{E}}
\newcommand{\curH}{\mathscr{H}}
\newcommand{\curJ}{\mathscr{J}}
\newcommand{\curK}{\mathscr{K}}
\newcommand{\curN}{\mathscr{N}}
\newcommand{\curO}{\mathscr{O}}
\newcommand{\curQ}{\mathscr{Q}}
\newcommand{\curS}{\mathscr{S}}
\newcommand{\curT}{\mathscr{T}}
\newcommand{\curU}{\mathscr{U}}
\newcommand{\curV}{\mathscr{V}}
\newcommand{\curW}{\mathscr{W}}
\newcommand{\curZ}{\mathscr{Z}}
\newcommand{\curI}{\mathscr{I}}
\newcommand{\curB}{\mathscr{B}}
\newcommand{\curF}{\mathscr{F}}
\newcommand{\curG}{\mathscr{G}}
\newcommand{\curM}{\mathscr{M}}
\newcommand{\curL}{\mathscr{L}}
\newcommand{\curP}{\mathscr{P}}
\newcommand{\curR}{\mathscr{R}}
\newcommand{\curX}{\mathscr{X}}
\newcommand{\curY}{\mathscr{Y}}
\newcommand{\calA}{\mathcal{A}}
\newcommand{\calB}{\mathcal{B}}
\newcommand{\calC}{\mathcal{C}}
\newcommand{\calD}{\mathcal{D}}
\newcommand{\calE}{\mathcal{E}}
\newcommand{\calF}{\mathcal{F}}
\newcommand{\calG}{\mathcal{G}}
\newcommand{\calH}{\mathcal{H}}
\newcommand{\calI}{\mathcal{I}}
\newcommand{\calJ}{\mathcal{J}}
\newcommand{\calL}{\mathcal{L}}
\newcommand{\calK}{\mathcal{K}}
\newcommand{\calM}{\mathcal{M}}
\newcommand{\calN}{\mathcal{N}}
\newcommand{\calO}{\mathcal{O}}
\newcommand{\calP}{\mathcal{P}}
\newcommand{\calQ}{\mathcal{Q}}
\newcommand{\calR}{\mathcal{R}}
\newcommand{\calS}{\mathcal{S}}
\newcommand{\calT}{\mathcal{T}}
\newcommand{\calU}{\mathcal{U}}
\newcommand{\calV}{\mathcal{V}}
\newcommand{\calW}{\mathcal{W}}
\newcommand{\calX}{\mathcal{X}}
\newcommand{\calY}{\mathcal{Y}}
\newcommand{\calZ}{\mathcal{Z}}
\newcommand{\RA}{\Rightarrow}
\newcommand{\ra}{\rightarrow}
\newcommand{\fd}{\vspace{2.5mm}}
\newcommand{\ds}{\vspace{1mm}}
\setlength{\parindent}{0pt}
\begin{document}
These notes loosely follow Fern{\'a}ndez-Villaverde and Levintal (2018) ``Solution Methods for Models with Rare Disasters'' but omits several features, such as recursive preferences and disaster risk.

\section{Model}\label{sec:model}

\subsection{Household}

The model admits a representative agent, so I directly write households' problem as the representative agent's. The representative household solves, in the cashless limit,
\begin{align}\label{eq:hh objective}
  \max_{C_t, L_t, B_{t + 1}, X_t, K_{t + 1}} \E_0 \sum_{t = 0}^\infty \beta^t\exp(\eta_{\beta,t}) \left( \frac{C_t^{1 - \gamma}}{1 - \gamma} -  \varphi\exp(\eta_{L, t})\frac{L_t^{1 + \nu}}{1 + \nu}\right)
\end{align}
subject to the budget constraint
\begin{align}\label{eq:hh budget constraint}
  C_t + \frac{B_{t + 1}}{P_t} + X_t  \leq W_t L_t + R_{K, t} K_t + R_{t - 1} \frac{B_t}{P_t} + F_t + T_t,
\end{align}
where $C_t$ is consumption, $B_t$ nominal bonds, $X_t$ investment, $W_{N, t}$ the real wage, $L_t$ labor, $R_{KN, t}$ the gross real rental rate on capital,
$K_t$ capital, $R_t$ the gross nominal interest rate on bonds, $F_t$ real profits from firms, and $T_t$ real lump-sum transfers from the government.
The price of the final consumption good is $P_t$. Markets are assumed complete, but securities are in zero net supply. Because there is a representative agent, I may omit the Arrow securities from the budget constraint. Our notation treats
$B_t$ and $K_t$ as pre-determined at time $t$.

Investment for capital follows the law of motion
\begin{align}\label{eq:invst eqn}
  K_{t + 1} & = (1 - \delta) K_t + \exp(\eta_{K, t})\left(1 - \Phi\left[\frac{X_t}{X_{t - 1}}\right]\right)X_t,
\end{align}
where the investment function takes the functional form
\begin{align*}
  \Phi\left[\frac{X_t}{X_{t - 1}}\right] = \frac{\chi}{2}\left(\frac{X_t}{X_{t - 1}} \right)^2.
\end{align*}




The Lagrangian for the household is
\begin{align*}
\calL & = \E_0 \sum_{t = 0}^\infty \beta^t\exp(\eta_{\beta, t})\left[\frac{C_t^{1 - \gamma}}{1 - \gamma} - \varphi\exp(\eta_{L, t}) \frac{L_t^{1 + \nu}}{1 + \nu}\right]\\
        &\quad + \E_0 \sum_{t = 0}^\infty \beta^t\exp(\eta_{\beta, t})\lambda_t\left[ W_{t} L_t + R_{K, t} K_t + R_{t - 1}\frac{B_t}{P_t} + F_t + T_t - C_t -\frac{ B_{t + 1}}{P_t} - X_t\right]\\
        &\quad + \E_0 \sum_{t = 0}^\infty \beta^t\exp(\eta_{\beta, t})\lambda_t Q_t\left[(1 - \delta)K_t +\exp( \eta_{K, t})\left(1 - \Phi\left[\frac{X_t}{X_{t - 1}}\right]\right)X_t - K_{t + 1}\right],
\end{align*}
which implies first-order conditions
\begin{align*}
  0 & = C_t^{- \gamma} - \lambda_t\\
  0 & = -\varphi\exp(\eta_{L, t}) L_t^\nu + \lambda_t W_{t}\\
  0 & = -\exp(\eta_{\beta, t})\lambda_t + \beta \E_t[\exp(\eta_{\beta, t + 1})\lambda_{t + 1}R_t]\\
  0 & = -\exp(\eta_{\beta, t})\lambda_t + \exp(\eta_{\beta, t})\lambda_tQ_t \exp(\eta_{K, t})\left(1 - \Phi\left[\frac{X_t}{X_{t - 1}}\right]\right)\\
    &\quad - \exp(\eta_{\beta, t})\lambda_tQ_t\exp(\eta_{K, t})\Phi'\left[\frac{X_t}{X_{t - 1}}\right]\frac{X_t}{X_{t - 1}}\\
    &\quad  - \beta\E_t[\exp(\eta_{\beta, t + 1})\lambda_{t + 1}Q_{t + 1}\exp(\eta_{K, t + 1})\Phi'\left[\frac{X_{t + 1}}{X_t}\right]\left(-\frac{X_{t + 1}}{X_t^2}\right)X_{t + 1}\\
  0 & =  \beta\E_t[\exp(\eta_{\beta, t + 1}) \lambda_{t + 1} (R_{K, t + 1} + Q_{t + 1}(1 - \delta))] - \exp(\eta_{\beta, t})\lambda_{t} Q_{t}
\end{align*}
The first two equations can be combined by isolating $\lambda_t$, which obtains the intratemporal consumption-labor condition
\begin{align*}
    C_t^{-\gamma} W_t & = \varphi\exp(\eta_{L, t}) L_t^\nu,
\end{align*}
Using $\lambda_t = C_t^{-\gamma}$ and defining the gross inflation rate $\Pi_t \equiv P_{t} / P_{t - 1}$, I can obtain the Euler equation for households
\begin{align*}
  \exp(\eta_{\beta, t})\frac{C_t^{-\gamma}}{P_t} & = \beta \E_t\left[\exp(\eta_{\beta, t + 1})\frac{C_{t + 1}^{-\gamma}}{P_{t + 1}}R_t\right]\\
  1 & = \beta \E_t\left[\frac{\exp(\eta_{\beta, t + 1})}{ \exp(\eta_{\beta, t})}\frac{C_{t + 1}^{-\gamma}}{ C_t^{-\gamma}}\frac{R_t}{\Pi_{t + 1}}\right].
\end{align*}
I can further simplify the Euler equation by defining the (real) stochastic discount factor
\begin{align*}
  M_{t + 1} & = \beta\frac{\exp(\eta_{\beta, t + 1})}{\exp(\eta_{\beta, t})}\frac{C_{t + 1}^{-\gamma}}{C_t^{-\gamma}}
\end{align*}
After dividing through by $\exp(\eta_{\beta, t})\lambda_t$ and re-arranging, the investment condition becomes
\begin{align*}
  1 & = \exp(\eta_{K, t}) Q_t \left(1 - \Phi\left[\frac{X_t}{X_{t - 1}}\right]\right) - \exp(\eta_{K, t})Q_t\Phi'\left[\frac{X_t}{X_{t - 1}}\right]\frac{X_t}{X_{t - 1}}\\
    &\quad + \E_t\left[M_{t + 1}\exp(\eta_{K, t + 1})Q_{t + 1} \Phi'\left[\frac{X_{t + 1}}{X_t}\right]\frac{X_{t + 1}^2}{X_t^2}\right]\\
    & =  \exp(\eta_{K, t}) Q_t \left(\left(1 - \Phi\left[\frac{X_t}{X_{t - 1}}\right]\right) - \Phi'\left[\frac{X_t}{X_{t - 1}}\right]\frac{X_t}{X_{t - 1}}\right) \\
    &\quad + \E_t\left[M_{t + 1}\exp(\eta_{K, t + 1})Q_{t + 1} \Phi'\left[\frac{X_{t + 1}}{X_t}\right]\frac{X_{t + 1}^2}{X_t^2}\right].
\end{align*}
Finally, after dividing through by $\exp(\eta_{\beta, t})\lambda_t$, the first-order condition for next-period capital is
\begin{align*}
  Q_t & = \E_t[M_{t + 1} (R_{K, t + 1} + Q_{t + 1}(1  - \delta))]
\end{align*}
In summary, households' optimality conditions are
\begin{align}
  \label{eq:consumption labor}
    C_t^{-\gamma} W_t & = \varphi\exp(\eta_{L, t}) L_t^\nu,\\
  \label{eq:stochastic discount factor}
  M_{t + 1} & = \beta\frac{\exp(\eta_{\beta, t + 1})}{\exp(\eta_{\beta, t})}\frac{C_{t + 1}^{-\gamma}}{C_t^{-\gamma}},\\
  \label{eq:euler eqn}
  1 & = \beta \E_t\left[M_{t + 1}\frac{R_t}{\Pi_{t + 1}}\right],\\
  \label{eq:tobins q}
  \begin{split}
    1  & =  \exp(\eta_{K, t}) Q_t \left(\left(1 - \Phi\left[\frac{X_t}{X_{t - 1}}\right]\right) - \Phi'\left[\frac{X_t}{X_{t - 1}}\right]\frac{X_t}{X_{t - 1}}\right) \\
    &\quad + \E_t\left[M_{t + 1}\exp(\eta_{K, t + 1})Q_{t + 1} \Phi'\left[\frac{X_{t + 1}}{X_t}\right]\frac{X_{t + 1}^2}{X_t^2}\right],
  \end{split}\\
  \label{eq:capital asset pricing}
  Q_t & = \E_t[M_{t + 1} (R_{K, t + 1} + Q_{t + 1}(1  - \delta))].
\end{align}



\subsection{Production}

\paragraph{Final Producers}

There is a representative final goods firm which sells consumption goods in a competitive market.
It aggregates intermediate goods using the CES technology
\begin{align*}
  Y_t & = \left(\int_0^1 Y_t(j)^{ \frac{\epsilon - 1}{\epsilon}}\right)^{\frac{\epsilon}{\epsilon - 1}}
\end{align*}
where $\epsilon > 1$ so that inputs are substitutes. Profit maximization for the final good firm is
\begin{align*}
  \max_{Y_t(j)} P_t\left(\int_0^1 Y_t(j)^{ \frac{\epsilon - 1}{\epsilon}}\right)^{\frac{\epsilon}{\epsilon - 1}} - \int_0^1 P_t(j) Y_t(j)\, dj.
\end{align*}
The FOC for $Y_t(j)$ is
\begin{align*}
  0 & = P_t\frac{\epsilon}{\epsilon - 1}\left(\int_0^1 Y_t(j)^{\frac{\epsilon}{\epsilon - 1}}\right)^{\frac{1}{\epsilon - 1}}\frac{\epsilon - 1}{\epsilon} Y_t(j)^{-\frac{1}{\epsilon}} - P_t(j)\\
  0 & = \left(\int_0^1 Y_t(j)^{\frac{\epsilon}{\epsilon - 1}}\right)^{\frac{1}{\epsilon - 1}}Y_t(j)^{-\frac{1}{\epsilon}} - \frac{P_t(j)}{P_t}\\
  0 & = \left(\int_0^1 Y_t(j)^{\frac{\epsilon}{\epsilon - 1}}\right)^{-\frac{\epsilon}{\epsilon - 1}}Y_t(j) - \left(\frac{P_t(j)}{P_t}\right)^{-\epsilon}\\
  Y_t(j) & =  \left(\frac{P_t(j)}{P_t}\right)^{-\epsilon} Y_t.
\end{align*}
Plugging this quantity into the identity
\begin{align*}
  P_tY_t & = \int_0^1 P_t(j) Y_t(j)\, dj
\end{align*}
and simplifying yields the price index
\begin{align*}
  P_t & = \left(\int_0^1 P_t(j)^{1 - \epsilon}\, dj\right)^{\frac{1}{1 - \epsilon}}.
\end{align*}

\paragraph{Intermediate Producers}

Intermediate goods are producing according to the Cobb-Douglas technology
\begin{align*}
  Y_t(j) = A_t K_t^\alpha(j)L_t^{1 - \alpha}(j).
\end{align*}
Intermediate producers minimize cost subject to the constraint of meeting demand and Calvo price rigidities. Formally,
\begin{align*}
  \min_{K_t(j), L_t(j)} R_{K, t} K_t(j) + W_t L_t(j)\quad\quad\quad\text{s.t.}\quad\quad\quad A_t K_t^\alpha(j) L_t^{1 - \alpha}(j) \geq \left(\frac{P_t(j)}{P_t}\right)^{-\epsilon}Y_t.
\end{align*}
The RHS of the inequality constraint is the demand from final goods producers for intermediate $j$. The Lagrangian is
\begin{align*}
  \calL = R_{K, t} K_t(j) + W_t L_t(j) + MC_t(j)\left(\left(\frac{P_t(j)}{P_t}\right)^{-\epsilon}Y_t - A_t K_t^\alpha(j) L_t^{1 - \alpha}(j)\right),
\end{align*}
so the first-order conditions are
\begin{align*}
  0 & = R_{K, t} - MC_t(j) \alpha A_t \left(\frac{L_t(j)}{K_t(j)}\right)^{1 - \alpha}\\
  0 & = W_t - MC_t(j) (1-\alpha) A_t \left(\frac{K_t(j)}{L_t(j)}\right)^{\alpha},
\end{align*}
hence the optimal capital-labor ratio satisfies
\begin{align*}
  \frac{R_{K, t}}{\alpha A_t(K_t(j)/L_t(j))^{\alpha - 1}} & = \frac{W_t}{(1-\alpha) A_t(K_t(j)/L_t(j))^{ \alpha}}\\
  \frac{K_t(j)}{L_t(j)} & =\frac{\alpha}{1 - \alpha} \frac{W_t}{R_{K, t}}.
\end{align*}
Since the RHS does not vary with $j$, all firms choose the same capital-labor ratio. Given this optimal ratio, the marginal cost satisfies
\begin{align*}
  MC_t & = \frac{R_{K, t}}{\alpha A_t}\left(\frac{K_t}{L_t}\right)^{1 - \alpha}\\
       & =  \frac{R_{K, t}}{\alpha A_t}\left(\frac{\alpha}{1 - \alpha} \frac{W_t}{R_{K, t}}\right)^{1 - \alpha}\\
       & =  \left(\frac{1}{1 - \alpha}\right)^{1 - \alpha}\left(\frac{1}{\alpha}\right)^{\alpha}\frac{W_t^{1 - \alpha}R_{K, t}^{\alpha}}{ A_t}.
\end{align*}
It follows that
\begin{align*}
  R_{K, t}K_t + W_tL_t & = \left(\frac{R_{K, t}}{A_t}\left(\frac{K_t}{L_t}\right)^{1 - \alpha} + \frac{W_t}{A_t}\left(\frac{L_t}{K_t}\right)^{\alpha}\right)(A_tK_t^\alpha L_t^{1 - \alpha})\\
                       & = \left(\alpha MC_t + (1 - \alpha)MC_t\right)Y_t(j) = MC_t Y_t(j).
\end{align*}
Therefore, (real) profits for an intermediate producer become
\begin{align*}
  F_t(j) = \frac{P_t(j)}{P_t}Y_t(j) - MC_t Y_t(j).
\end{align*}
In addition to the capital-labor choice, firms also have the chance to reset prices in every period with probability $1 - \theta$. This problem can be
written as
\begin{align*}
  \max_{P_t(j)} \E_t\sum_{s = 0}^\infty(\beta \theta)^s\frac{\exp(\eta_{\beta, t + s})}{\exp(\eta_{\beta, t})}\frac{u'(C_{t + s})}{u'(C_t)}\left(\frac{P_t(j)}{P_{t + s}}\left(\frac{P_t(j)}{P_{t + s}}\right)^{-\epsilon}Y_{t + s} - mc_{t + s}\left(\frac{P_t(j)}{P_{t + s}}\right)^{-\epsilon}Y_{t + s}\right),
\end{align*}
where I have imposed that intermediate output equals demand. The first-order condition is
\begin{align*}
0 & =  (1 - \epsilon)P_t(j)^{-\epsilon}\E_t\sum_{s = 0}^\infty (\beta\theta)^s \frac{\exp(\eta_{\beta, t + s})}{\exp(\eta_{\beta, t})}\frac{u'(C_{t + s})}{u'(C_t)}(P_{t + s})^{-(1 - \epsilon)}Y_{t + s}\\
  &\quad + \epsilon P_t(j)^{-\epsilon - 1}\E_t\sum_{s = 0}^\infty (\beta\theta)^s \frac{\exp(\eta_{\beta, t + s})}{\exp(\eta_{\beta, t})}\frac{u'(C_{t + s})}{u'(C_t)} mc_{t + s}P_{t + s}^{\epsilon}Y_{t + s}
\end{align*}
Divide by $P_t(j)^{-\epsilon} / (\exp(\eta_{\beta, t})u'(C_t))$, apply the abuse of notation that $\prod_{u = 1}^0 \Pi_{t + u} = 1$, and re-arrange to obtain
\begin{align*}
  P_t(j) & = \frac{\epsilon}{\epsilon - 1}\frac{\E_t\sum_{s = 0}^\infty (\beta\theta)^s \exp(\eta_{\beta, t + s})u'(C_{t + s}) mc_{t + s}P_{t + s}^{\epsilon}Y_{t + s}}{\E_t\sum_{s = 0}^\infty (\beta\theta)^s \exp(\eta_{\beta, t + s})u'(C_{t + s})P_{t + s}^{\epsilon - 1}Y_{t + s}}\\
         & = \frac{\epsilon}{\epsilon - 1}\frac{\E_t\sum_{s = 0}^\infty (\beta\theta)^s \exp(\eta_{\beta, t + s})u'(C_{t + s}) mc_{t + s}P_t^\epsilon\left(\prod_{u = 1}^s\Pi_{t + u}\right)^{\epsilon}Y_{t + s}}{\E_t\sum_{s = 0}^\infty (\beta\theta)^s \exp(\eta_{\beta, t + s})u'(C_{t + s})P_t^{\epsilon - 1}\left(\prod_{u = 1}^s\Pi_{t + u}\right)^{\epsilon - 1}Y_{t + s}}\\
  \frac{P_t(j)}{P_t} & = \frac{\epsilon}{\epsilon - 1}\frac{\E_t\sum_{s = 0}^\infty (\beta\theta)^s \exp(\eta_{\beta, t + s})u'(C_{t + s}) mc_{t + s}\left(\prod_{u = 1}^s\Pi_{t + u}\right)^{\epsilon}Y_{t + s}}{\E_t\sum_{s = 0}^\infty (\beta\theta)^s \exp(\eta_{\beta, t + s})u'(C_{t + s})\left(\prod_{u = 1}^s\Pi_{t + u}\right)^{\epsilon - 1}Y_{t + s}}.
\end{align*}
This expression gives the optimal (real) reset price $P_t^* \equiv P_t(j) / P_t $ (note that the RHS does not depend on $j$).
Define
\begin{align*}
  S_{1, t} & = \E_t\sum_{s = 0}^\infty  (\beta\theta)^s \exp(\eta_{\beta, t + s})u'(C_{t + s}) mc_{t + s}Y_{t + s}\left( \prod_{u = 1}^s\Pi_{t + s}\right)^{\epsilon},\\
  S_{2, t}  & = \E_t\sum_{s = 0}^\infty (\beta\theta)^s \exp(\eta_{\beta, t + s})u'(C_{t + s})Y_{t + s}\left(\prod_{u = 1}^s\Pi_{t + s}\right)^{\epsilon - 1}.
\end{align*}
Using these definitions, I may write the optimal reset price more compactly as
\begin{align*}
  P_t^* & = \frac{\epsilon}{\epsilon - 1}\frac{S_{1, t}}{S_{2, t}}
\end{align*}
where $S_{1, t}$ and $S_{2, t}$ satisfy the recursions
\begin{align*}
  S_{1, t} & = \exp(\eta_{\beta, t})u'(C_t) MC_t Y_t + \theta \beta \E_t\Pi_{t + s}^\epsilon S_{1, t + 1}\\
  S_{2, t} & = \exp(\eta_{\beta, t})u'(C_t) Y_t + \theta \beta \E_t \Pi_{t + s}^{\epsilon - 1}S_{2, t + 1}.
\end{align*}
These recursions can be further rewritten as
\begin{align*}
  \frac{S_{1, t}}{\exp(\eta_{\beta, t})u'(C_t)} & =  MC_t Y_t + \theta \beta \E_t\left[\frac{\exp(\eta_{\beta, t + 1}) u'(C_{t + 1})}{\exp(\eta_{\beta, t}) u'(C_t)}\Pi_{t + s}^\epsilon \frac{S_{1, t + 1}}{\exp(\eta_{\beta, t + 1})u'(C_{t + 1})}\right]\\
  \frac{S_{2, t}}{\exp(\eta_{\beta, t})u'(C_t)} & =  Y_t + \theta \beta \E_t\left[\frac{\exp(\eta_{\beta, t + 1}) u'(C_{t + 1})}{\exp(\eta_{\beta, t}) u'(C_t)}\Pi_{t + s}^{\epsilon - 1} \frac{S_{2, t + 1}}{\exp(\eta_{\beta, t + 1})u'(C_{t + 1})}\right],
\end{align*}
By defining $\tilde{S}_{1, t} \equiv S_{1, t}/ (\exp(\eta_{\beta, t})u'(C_t))$ and $\tilde{S}_{2, t} \equiv S_{2, t}/ (\exp(\eta_{\beta, t})u'(C_t))$, I can simplify these recursions into the form I use for the numerical solution.\\

From this section, we obtain the following five equilibrium conditions:
\begin{align}
  \label{eq:mc soln}
  MC_t & =  \left(\frac{1}{1 - \alpha}\right)^{1 - \alpha}\left(\frac{1}{\alpha}\right)^{\alpha}\frac{W_t^{1 - \alpha}R_{K, t}^{\alpha}}{ A_t},\\
  \label{eq:optimal capital labor ratio}
  \frac{K_t}{L_t} & =\frac{\alpha}{1 - \alpha} \frac{W_t}{R_{K, t}},\\
  \label{eq:real optimal reset price}
  P_t^* & = \frac{\epsilon}{\epsilon - 1}\frac{\tilde{S}_{1, t}}{\tilde{S}_{2, t}},\\
  \label{eq:numerator recursion}
  \tilde{S}_{1, t} & = MC_t Y_t + \theta\beta\E_t[M_{t + 1} \Pi_{t + s}^\epsilon \tilde{S}_{1, t + 1}],\\
  \label{eq:denominator recursion}
  \tilde{S}_{2, t} & =  Y_t + \theta\beta\E_t[M_{t + 1} \Pi_{t + s}^{\epsilon - 1} \tilde{S}_{2, t + 1}].
\end{align}

\subsection{Monetary Policy}
To close the model, I specify the monetary policy rule as the following Taylor rule
\begin{align}\label{eq:taylor rule}
  \frac{R_t}{R} & =  \left(\frac{R_{t - 1}}{R}\right)^{\phi_R}\left(\left(\frac{\Pi_t}{\Pi}\right)^{\phi_\Pi}\left(\frac{Y_t}{Y_{t - 1}}\right)^{\phi_Y}\right)^{1 - \phi_R}\exp(\eta_{R, t})
\end{align}
Any proceeds from monetary policy are distributed as lump sum to the representative household.

\subsection{Aggregation}
The price level is currently characterized as the integral
\begin{align*}
  P_t^{1 - \epsilon} = \int_0^1 P_t(j)^{1 - \epsilon}\, dj.
\end{align*}
To represent the model entirely in terms of aggregates, notice that, without loss of generality, we may re-order the fraction $\phi$ of firms which cannot reset prices to the top of the interval so that
\begin{align*}
  P_t^{1 - \epsilon} = (1 - \phi)(P_t^*)^{1 - \epsilon} +  \int_{1 - \phi}^1 P_{t - 1}(j)^{1 - \epsilon}\, dj.
\end{align*}
The latter term can be further simplified under the law of large numbers assumption that a positive measure of firms which cannot change their price
still comprise a representative sample of all firms, yielding
\begin{align*}
  P_t^{1 - \epsilon} = (1 - \phi)(P_t^*)^{1 - \epsilon} +  \phi\int_0^1 P_{t - 1}(j)^{1 - \epsilon}\, dj = (1 - \phi)(P_t^*)^{1 - \epsilon} +  \phi P_{t - 1}^{1 - \epsilon}.
\end{align*}
Dividing by $P_{t - 1}^{1 - \epsilon}$ implies
\begin{align}\label{eq:inflation from optimal reset price}
  \Pi_t^{ 1 - \epsilon} & = (1 - \phi) (P_t^*\Pi_t)^{1 - \epsilon} + \phi.
\end{align}
The price dispersion term can similarly be re-written in terms of aggregates by distinguishing which firms get to change prices.
\begin{align*}
  v_t & = \int_0^{1 - \phi}\left(P_t^*\right)^{ - \epsilon}\, dj + \int_{1 - \phi}^1 \left(\frac{P_{t - 1}(j)}{P_t}\right)^{ - \epsilon}\, dj\\
      & = \int_0^{1 - \phi}\left(P_t^*\Pi_t\right)^{ - \epsilon}\left(\frac{1}{\Pi_t}\right)^{ - \epsilon}\, dj + \int_{1 - \phi}^1 \left(\frac{P_{t - 1}(j)}{P_{t - 1}}\right)^{ - \epsilon}\left(\frac{P_{t - 1}}{P_t}\right)^{ - \epsilon}\, dj\\
      & = (1 - \phi) (P_t^*\Pi_t)^{-\epsilon} \Pi_t^{\epsilon}  + \Pi_t^{\epsilon} \int_{1 - \phi}^1 \left(\frac{P_{t - 1}(j)}{P_{t - 1}}\right)^{ - \epsilon}\,dj.
\end{align*}
By invoking the law of large assumptions applied to any positive measure subset of firms, we must have
\begin{align*}
  \int_{1 - \phi}^1 \left(\frac{P_{t - 1}(j)}{P_{t - 1}}\right)^{ - \epsilon}\,dj & = \phi\int_0^1 \left(\frac{P_{t - 1}(j)}{P_{t - 1}}\right)^{ - \epsilon}\,dj = \phi v_{t - 1}.
\end{align*}
Thus, we acquire
\begin{align}\label{eq:price dispersion evol}
  v_t & = \Pi_t^{\epsilon}((1 - \phi) (P_t^* \Pi_t)^{\epsilon} + \phi v_{t - 1})
\end{align}


\subsection{Equilibrium}
To close the model, I need to specify the functional form for investment, aggregate shocks, and market-clearing conditions.\\

Following the literature, I assume quadratic adjustment investment costs so that
\begin{align}\label{eq:invst adjst cost fnct}
  \Phi\left[\frac{X_t}{X_{t - 1}}\right] & = \frac{\chi}{2}\left(\frac{X_t}{X_{t - 1}}\right)^2,
\end{align}
which has first derivative w.r.t. $X_t / X_{t - 1}$
\begin{align}\label{eq:invst adjst cost fnct first deriv}
    \Phi'\left[\frac{X_t}{X_{t - 1}}\right] & = \chi\left(\frac{X_t}{X_{t - 1}}\right).
\end{align}

There are five shocks in the model: $\eta_{A, t}$, $\eta_{\beta, t}$, $\eta_{L, t}$, $\eta_{K, t}$, and $\eta_{R, t}$. Without loss of generality, I assume all shocks follow AR(1) processes with persistence $\rho_i$ and standard deviation $\sigma_i$.\\

Markets must clear for capital, labor, bonds, final goods, and intermediate goods, . The first three markets clear as a consequence of optimality conditions and the assumption that bonds have zero net supply. To clear the market for final goods, we set aggregate consumption demand $C_t$ equal to aggregate supply $Y_t$, which satisfies
\begin{align*}
\int_0^1 A_t K_t^\alpha L_t^{1 - \alpha}\, dj & = \int_0^1\left(\frac{P_t(j)}{P_t}\right)^{-\epsilon} Y_t\, dj\\
  A_tK_t^\alpha L_t^{1 - \alpha} & = Y_t \int_0^1\left(\frac{P_t(j)}{P_t}\right)^{-\epsilon} \, dj = v_t Y_t.
\end{align*}
Re-arranging yields the consumption market-clearing condition
\begin{align}\label{eq:consumption market clearing}
  C_t & = \frac{A_t K_t^{\alpha}L_t^{1 - \alpha}}{v_t}.
\end{align}
It can be shown that $v_t \geq 1$ by applying Jensen's inequality.


All together, the full set of endogenous equilibrium conditions are
\begin{align}
  \label{eq:consumption labor eqm}
    C_t^{-\gamma} W_t & = \varphi\exp(\eta_{L, t}) L_t^\nu,\\
  \label{eq:stochastic discount factor}
  M_{t + 1} & = \beta\frac{\exp(\eta_{\beta, t + 1})}{\exp(\eta_{\beta, t})}\frac{C_{t + 1}^{-\gamma}}{C_t^{-\gamma}},\\
  \label{eq:euler eqn eqm}
  1 & = \beta \E_t\left[M_{t + 1}\frac{R_t}{\Pi_{t + 1}}\right],\\
  \label{eq:tobins q eqm}
    1  & =  \exp(\eta_{K, t}) Q_t \left(1 - \frac{3\chi}{2}\left(\frac{X_t}{X_{t - 1}}\right)^2\right)  + \E_t\left[M_{t + 1}\exp(\eta_{K, t + 1})Q_{t + 1} \chi \left(\frac{X_{t + 1}}{X_t}\right)^3\right],\\
  \label{eq:capital asset pricing eqm}
  Q_t & = \E_t[M_{t + 1} (R_{K, t + 1} + Q_{t + 1}(1  - \delta))],\\
  \label{eq:mc soln eqm}
  MC_t & =  \left(\frac{1}{1 - \alpha}\right)^{1 - \alpha}\left(\frac{1}{\alpha}\right)^{\alpha}\frac{W_t^{1 - \alpha}R_{K, t}^{\alpha}}{ A_t},\\
  \label{eq:optimal capital labor ratio eqm}
  \frac{K_t}{L_t} & =\frac{\alpha}{1 - \alpha} \frac{W_t}{R_{K, t}},\\
  \label{eq:real optimal reset price eqm}
  P_t^* & = \frac{\epsilon}{\epsilon - 1}\frac{\tilde{S}_{1, t}}{\tilde{S}_{2, t}},\\
  \label{eq:numerator recursion eqm}
  \tilde{S}_{1, t} & = MC_t Y_t + \theta\beta\E_t[M_{t + 1} \Pi_{t + s}^\epsilon \tilde{S}_{1, t + 1}],\\
  \label{eq:denominator recursion eqm}
  \tilde{S}_{2, t} & =  Y_t + \theta\beta\E_t[M_{t + 1} \Pi_{t + s}^{\epsilon - 1} \tilde{S}_{2, t + 1}],\\
  \label{eq:inflation from optimal reset price eqm}
  \Pi_t^{ 1 - \epsilon} & = (1 - \phi) (P_t^*\Pi_t)^{1 - \epsilon} + \phi,\\
  \label{eq:price dispersion evol eqm}
  v_t & = \Pi_t^{\epsilon}((1 - \phi) (P_t^* \Pi_t)^{\epsilon} + \phi v_{t - 1}),\\
  \label{eq:taylor rule eqm}
  \frac{R_t}{R} & =  \left(\frac{R_{t - 1}}{R}\right)^{\phi_R}\left(\left(\frac{\Pi_t}{\Pi}\right)^{\phi_\Pi}\left(\frac{Y_t}{Y_{t - 1}}\right)^{\phi_Y}\right)^{1 - \phi_R}\exp(\eta_{R, t}),\\
  \label{eq:consumption market clearing eqm}
  C_t & = \frac{A_t K_t^{\alpha}L_t^{1 - \alpha}}{v_t},
\end{align}
as well as the law of motion for capital
\begin{align}\label{eq:law of motion capital eqm}
  K_{t + 1} & = (1 - \delta)K_t + \exp(\eta_{K, t})\left(1 - \frac{\chi}{2}\left(\frac{X_t}{X_{t - 1}}\right)^2\right)X_t
\end{align}
and the five exogenous processes
\begin{align}
  \eta_{\beta, t + 1} & = \rho_\beta\eta_{\beta, t} + \sigma_\beta \varepsilon_{\beta, t + 1},\\
  \eta_{L, t + 1} & = \rho_L\eta_{L, t} + \sigma_L \varepsilon_{L, t + 1},\\
  \eta_{A, t + 1} & = \rho_A\eta_{A, t} + \sigma_A \varepsilon_{A, t + 1},\\
  \eta_{K, t + 1} & = \rho_K\eta_{K, t} + \sigma_K \varepsilon_{K, t + 1},\\
  \eta_{R, t + 1} & = \rho_R\eta_{R, t} + \sigma_R \varepsilon_{R, t + 1}.
\end{align}

\section{Risk-Adjusted Linearization}\label{sec:ral}

We now proceed to converting the equilibrium conditions into a suitable form for a risk-adjusted linearization. The system should conform to the representation
\begin{align*}
  0 & = \log \E_t\left[\exp\left(\xi(z_t, y_t) + \Gamma_5 z_{t + 1} + \Gamma_6 y_{t + 1}\right)\right]\\
  z_{t + 1} & = \mu(z_t, y_t) + \Lambda(z_t, y_t) (y_{t + 1} - \E_t y_{t + 1}) + \Sigma(z_t, y_t) \varepsilon_{t + 1},
\end{align*}
where $z_t$ are (predetermined) state variables and $y_t$ are (nondetermined) jump variables.

For the remainder of this section, lower case variables are the logs of previously upper case variables, and with a small abuse of notation, let $s_{1, t} = \log(\tilde{S}_{1, t})$ and $s_{2, t} = \log(\tilde{S}_{2, t})$. Additionally, let $r_{k, t} = \log(R_{K, t})$.

\fd

Equation (\ref{eq:consumption labor eqm}) becomes
\begin{align*}
  1 & = \varphi \exp(\eta_{L, t})\frac{L_t^\nu}{C_t^{-\gamma}W_t}\\
  0 & = \log \E_t\left[\exp\left(\underbrace{\log(\varphi) + \eta_{L, t} + \nu l_t - (- \gamma c_t + w_t)}_{\xi}\right)\right].
\end{align*}

Equation (\ref{eq:stochastic discount factor}) will not be used in the system of equations for the risk-adjusted linearization,
but it simplifies the other equations. Taking logs and re-arranging yields
\begin{align*}
  0 & =  \log(\beta) + \eta_{\beta, t + 1} + (-\gamma c_{t + 1}) - \eta_{\beta, t} - (-\gamma c_t) - m_{t + 1}\\
    & = \underbrace{\log(\beta) -\eta_{\beta, t} + \gamma c_t}_{\xi} + \underbrace{\eta_{\beta, t + 1} - \gamma c_{t + 1} - m_{t + 1}}_{\text{forward-looking}}.
\end{align*}

Equation (\ref{eq:euler eqn eqm}) becomes
\begin{align*}
  0 & = \log\E_t\left[\exp\left(\underbrace{\log(\beta) + r_t}_{\xi} + \underbrace{m_{t + 1} - \Pi_{t + 1}}_{\text{forward-looking}}  \right)\right].
\end{align*}

Equation (\ref{eq:tobins q eqm}) becomes
\begin{align*}
  & 1 - \exp(\eta_{K, t})Q_t\left(1 - \frac{3\chi}{2}\left(\frac{X_t}{X_{t - 1}}\right)^2\right)\\
  & = \E_t\left[M_{t + 1}\exp(\eta_{K, t + 1})Q_{t + 1}\chi\left(\frac{X_{t + 1}}{X_t}\right)^3\right]\\
  & \log\left(1 - \exp(\eta_{K, t} + q_t)\left(1 - \frac{3\chi}{2}\exp(2x_t - 2x_{t - 1})\right)\right)\\
  & = \log\E_t\left[\exp\left(m_{t + 1} + \eta_{K, t + 1} + q_{t + 1} + \log(\chi) + 3 x_{t + 1} - 3 x_t\right)\right]\\
  0 & = \log\E_t\left[\frac{\exp\left(m_{t + 1} + \eta_{K, t + 1} + q_{t + 1} + \log(\chi) + 3 x_{t + 1} - 3 x_t\right)}{\exp\left(\log\left(1 - \exp(\eta_{K, t} + q_t)\left(1 - \frac{3\chi}{2}\exp(2x_t - 2x_{t - 1})\right)\right)\right)}\right].
\end{align*}
Re-arranging yields
\begin{align*}
  0 & = \log\E_t\left[\exp\left(\underbrace{\log(\chi)- 3x_t - \log\left(1 - \exp(\eta_{K, t} + q_t)\left(1 - \frac{3\chi}{2}\exp(2x_t - 2x_{t - 1})\right)\right)}_{\xi}\right)\right]\\
    &\quad + \log \E_t\left[\exp(\underbrace{m_{t + 1} + \eta_{K, t + 1} + q_{t + 1} + 3x_{t + 1}}_{\text{forward-looking}})\right].
\end{align*}

Equation (\ref{eq:capital asset pricing eqm}) becomes
\begin{align*}
  1 & = \E_t\left[\frac{M_{t + 1} (R_{K, t + 1} + Q_{t + 1}(1 - \delta))}{Q_t}\right]\\
  0 & = \log\E_t\left[\exp\left(m_{t + 1} + \log(\exp(r_{k, t + 1}) + (1-\delta)\exp(q_{t + 1})) - q_t\right)\right].
\end{align*}
This equation is not linear in the forward-looking variables, we define a new variable
\begin{align}\label{eq:Omega defn}
  N_t = R_{K, t} + Q_t(1 - \delta),
\end{align}
so that we may represent (\ref{eq:capital asset pricing eqm}) in the required form for risk-adjusted linearizations as the following two equations:
\begin{align*}
  0 & = \log\E_t\left[\exp\left(\underbrace{-q_t}_{\xi} + \underbrace{m_{t + 1} + n_{t + 1}}_{\text{forward-looking}}\right)\right],\\
  0 & = \log\E_t\left[\exp\left(\underbrace{n_t - \log(\exp(r_{k, t}) +  (1 - \delta)\exp(q_t)))}_{\xi} \right)\right],\\
\end{align*}

Equation (\ref{eq:mc soln eqm}) becomes
\begin{align*}
  0 & = \log\E_t\left[\exp\left(\underbrace{(1 - \alpha)w_t + \alpha r_{k, t} - a_t -(1 - \alpha)\log(1 - \alpha) - \alpha \log(\alpha) - mc_t}_{\xi} \right)\right].
\end{align*}

Equation (\ref{eq:optimal capital labor ratio eqm}) becomes
\begin{align*}
  0 & = \log\E_t\left[\exp\left(\underbrace{k_t - l_t - \log\left(\frac{\alpha}{1 - \alpha}\right) - (w_t - r_{k, t})}_{\xi} \right)\right].
\end{align*}

Like the stochastic discount factor, equation (\ref{eq:real optimal reset price eqm}) will not be used in the system of equations, but it will be useful to simplify other equations. Taking logs yields
\begin{align*}
  p_t^* & = \log\left(\frac{\epsilon}{\epsilon - 1}\right) + s_{1, t} - s_{2, t}.
\end{align*}


The first equation becomes
\begin{align*}
  1 & = \beta \E_t\left[\frac{C_{t + 1}^{-\sigma}}{C_t^{-\sigma}}\frac{ 1 + i_t}{1 + \pi_{t + 1}}\right]\\
  0 & = \log\E_t\left[\exp\left(\log(\beta) -\sigma (c_{t + 1} - c_t) + \tilde{i}_t - \tilde{\pi}_{t + 1} \right)\right]\\
    & = \log\E_t\left[\exp\left(\underbrace{\log(\beta) + \sigma c_t + \tilde{i}_t}_{\xi} - \underbrace{\sigma c_{t + 1} - \tilde{\pi}_{t + 1}}_{\text{Forward-Looking}} \right)\right],
\end{align*}
where $c_t = \log(C_t)$, $\tilde{i}_t = \log(1 + i_t)$, and $\tilde{\pi}_t = \log(1 + \pi_{t + 1})$.

\fd

The second equation becomes

\fd

The third equation becomes
\begin{align*}
  1 & = \frac{w_t}{A_t MC_t}\\
  0 & = \log\E_t\left[\exp\left(\hat{w}_t - a_t - \tilde{mc}_t \right)\right].
\end{align*}

The fourth and fifth equation become
\begin{align*}
  0 & = \log\E_t\left[\exp\left(c_t - a_t - n_t + \hat{v}_t \right)\right].
\end{align*}

The sixth equation becomes
\begin{align*}
  0 & = \hat{v}_t - \epsilon \tilde{\pi}_t - \log((1 - \phi) \exp(\tilde{\pi}_t^*)^{ - \epsilon} + \phi \exp(\hat{v}_{t - 1})),
\end{align*}
where $\hat{v}_{t - 1}$ will be treated as an additional state variable, i.e. if $a_t = \hat{v}_t$ is a jump variable and $b_t = \hat{v}_{t - 1}$
is a state variable, then
\begin{align*}
  b_{t + 1} = a_t.
\end{align*}

The seventh equation becomes
\begin{align*}
  0 & = (1 - \epsilon)\tilde{\pi}_t - \log((1 - \phi) \exp(\tilde{\pi}_t^*)^{1 - \epsilon} + \phi).
\end{align*}

The eighth equation becomes
\begin{align*}
  0 & = \tilde{\pi}_t^* - \log\left(\frac{\epsilon}{\epsilon - 1}\right) - \tilde{\pi}_t - (\hat{x}_{1, t} - \hat{x}_{2, t})).
\end{align*}
By plugging this expression for $\tilde{\pi}_t^*$ into the previous two equations, we can also remove one more variable from the system.

\fd

The ninth and tenth equation become
\begin{align*}
  1 & = \E_t\left[\phi \beta\frac{x_{1, t + 1}(1 + \pi_{t + 1})}{x_{1, t} - C_t^{-\sigma} MC_t A_tN_t / v_t}\right]\\
  0 & = \log\E_t\left[\exp\left(\underbrace{\log(\phi) + \log(\beta)   - \log(\exp(\hat{x}_{1, t}) - \exp((1-\sigma) c_t + \tilde{mc}_t))}_{\xi} + \hat{x}_{1, t + 1} + \epsilon\tilde{\pi}_{t + 1}\right)\right]\\
  0 & = \log\E_t\left[\exp\left(\underbrace{\log(\phi) + \log(\beta)  - \log(\exp(\hat{x}_{2, t}) - \exp((1-\sigma) c_t))}_{\xi} + \hat{x}_{2, t + 1} + (\epsilon - 1)\tilde{\pi}_{t + 1}\right)\right],
\end{align*}
where the fact that $x_{1,t}$ and $x_{2, t}$ must both be positive implies $x_{1, t} - C_t^{-\sigma} MC_t Y_t$ and $x_{2, t} - C_t^{-\sigma}Y_t$ are both positive, as the expectations on the RHS are also both positive.

\fd

For the monetary policy rule, we use $\tilde{i}_{t - 1} \equiv \log(1 + i_{t - 1})$ and $\varepsilon_{i, t}$ as states and treat $i_t$ as a jump variable, hence
\begin{align*}
  \tilde{i}_t & = (1 - \rho_i) \tilde{i} + \rho_i \tilde{i}_{t - 1} + (1 - \rho_i)\phi_\pi (\tilde{\pi}_t - \tilde{\pi})  + \varepsilon_{i, t}.
\end{align*}
This formulation allows us to treat the policy rule as an expectational equation.

\fd

The above nine equations comprise the expectational equations. The following four equations comprise the states:
\begin{align*}
  a_{t + 1} & = \rho_a a_t + \varepsilon_{a, t + 1}\\
  \hat{v}_{(t - 1) + 1} & = \hat{v}_t\\
  \tilde{i}_{(t - 1) + 1} & = \tilde{i}_t\\
  \varepsilon_{i, t + 1} & = \varepsilon_{i, t + 1}
\end{align*}


To conclude, note that we have three forward difference equations. The first is the Euler equation, which can be expressed as

\end{document}